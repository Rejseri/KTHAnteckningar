\documentclass[12pt,a4paper]{article}
\usepackage[utf8]{}
\usepackage[T1]{fontenc}
\title{Språck sammanfattning}
\author{Jamal Brown}


\renewcommand{\baselinestretch}{0.2}

\setlength{\hoffset}{-18pt}         
\setlength{\oddsidemargin}{0pt} % Marge gauche sur pages impaires
\setlength{\evensidemargin}{0pt} % Marge gauche sur pages paires
\setlength{\marginparwidth}{54pt} % Largeur de note dans la marge
\setlength{\textwidth}{481pt} % Largeur de la zone de texte (17cm)
\setlength{\voffset}{-18pt} % Bon pour DOS
\setlength{\marginparsep}{7pt} % Séparation de la marge
\setlength{\topmargin}{0pt} % Pas de marge en haut
\setlength{\headheight}{13pt} % Haut de page
\setlength{\headsep}{4pt} % Entre le haut de page et le texte
\setlength{\footskip}{27pt} % Bas de page + séparation
\setlength{\textheight}{720pt} % Hauteur de la zone de texte (25cm)

\usepackage{natbib}
\usepackage{graphicx}
\usepackage{amssymb}
\usepackage{amsmath} % Required for the align* environment
\usepackage{float}
\usepackage{pgfplots} 
\usepackage{algorithm}
\usepackage{algpseudocode}
\pgfplotsset{compat=1.18}


\begin{document}
\section{Föreläsning 1}
\subsection{Vad är makroekonomi?}
Studerar ekonomi som en helhet, inte enskilda marknader.
På både lång och kort sikt.

\subsection{Tre modeller}
\begin{itemize}
    \item Mycket långsikt: Solowmodellen
    \item Långsikt: AS-AD modellen
    \item Kortsikt: IS-LM modellen
\end{itemize}

\subsection{AS--AD modellen}
Aggregerad efterfrågaekurva(AD)
\begin{itemize}
    \item Sammanvägning av alla efterfrågekurvor i ekonomin
    \item Lutar alltid nedåt
\end{itemize}

\noindent Aggregerat utbudskurva(AS)
\begin{itemize}
    \item Sammanvägning av alla utbudskurvor i ekonomin
    \item Beror på tidshorisonten
\end{itemize}

\subsection{Aggregerad utbud--mycket långsikt}
\begin{figure}[H]
    \centering
    \includegraphics[width=0.5\textwidth]{ASLong.png}
\end{figure}
\begin{itemize}
    \item Ackumulering av produktionsfaktorer
    \item Teknisk utveckling
    \item Studerar vad det är som får kurvan att röra sig åt höger
\end{itemize}

\subsection{Aggregerat utbud--långsikt}
\begin{figure}[H]
    \centering
    \includegraphics[width=0.5\textwidth]{ASMedium.png}
\end{figure}
\begin{itemize}
    \item Klassiska modellen
    \item Produktionsfaktorer uttnyttjas fullt ut
    \item Alltid i jämnvikt
\end{itemize}

\subsection{Aggregerat utbud--kortsikt}
\begin{figure}[H]
    \centering
    \includegraphics[width=0.5\textwidth]{ASShort.png}
\end{figure}
Keynesiansk teori:
\begin{itemize}
    \item Priser och löner är trögrörliga (konstanta)
    \item Produktionsfaktorer inte fullt utnyttjade
\end{itemize}

\newpage
\subsection{Dynamik på kort och långsikt}
Exempel: Expansiv penningpolitik
\begin{figure}[H]
    \centering
    \includegraphics[width=1\textwidth]{Dynamic.png}
\end{figure}

\noindent På kort sikt:
\begin{itemize}
    \item Ökar produktionen dvs. kvantiteten
    \item Priserna är konstanta
\end{itemize}

\noindent På lång sikt:
\begin{itemize}
    \item Priserna ökar
    \item Produktionen återgår till det ursprungliga
    \item Ingen effekt på produktionen
\end{itemize}

\noindent På medellång sikt:
\begin{figure}[H]
    \centering
    \includegraphics[width=0.5\textwidth]{asm.png}
\end{figure}
\begin{itemize}
    \item Företaget producerar mer än det naturliga utbudet
    \item Företaget höjer priserna
    \item Produkten återgår till det naturliga utbudet
    \item $A\rightarrow B \rightarrow C$
\end{itemize}

\subsection{BNP och real tillväxt, Sverige 1820--2018}
\begin{figure}[H]
    \centering
    \includegraphics[width=0.75\textwidth]{BNP.png}
\end{figure}

\begin{itemize}
    \item Brukar mätas som procentuell förändring av BNP
    \item Normalt 2--3\% per år
    \item Beror på faktoranvändning och produktivitet 
\end{itemize}

\subsection{BNP och potentiell BNP}
\begin{figure}[H]
    \centering
    \includegraphics[width=0.75\textwidth]{BNP2.png}
\end{figure}

\begin{itemize}
    \item Potentiell BNP: Om samtliga produktionsfaktorer används fullt ut på ett långsiktigt hållbart sätt
    \item Produktionsgap = BNP - Potentiell BNP
    \item Övernyttjande bidrar till inflation
    \item Undernyttjande bidrar till arbetslöshet
\end{itemize}

\subsection{Produktionsgap, Sverige 1945--2018}
\begin{figure}[H]
    \centering
    \includegraphics[width=0.75\textwidth]{BNP3.png}
\end{figure}

\begin{figure}[H]
    \centering
    \includegraphics[width=0.75\textwidth]{BNP4.png}
\end{figure}

\begin{itemize}
    \item Vi kan se att variationen i produktionsgapet blir större och större i nominala termer
    \item Men procentuellt är den i princip lika stor
\end{itemize}

\subsection{Konjunkturcykelns terminologi}
\begin{figure}[H]
    \centering
    \includegraphics[width=0.75\textwidth]{konjunktur.png}
\end{figure}
\begin{itemize}
    \item BNP över potentiell BNP: Högkonjunktur
    \item BNP under potentiell BNP: Lågkonjunktur
    \item BNP påväg upp: konjunkturuppgång
    \item BNP påväg ner: konjunkturnedgång
\end{itemize}

\subsection{Samband produktion och arbetslöshet}
\begin{figure}[H]
    \centering
    \includegraphics[width=0.75\textwidth]{BNP5.png}
\end{figure}

\begin{itemize}
    \item När produktionen ökar, minskar arbetslösheten
    \item När produktionen minskar, ökar arbetslösheten
\end{itemize}

\subsection{AS och Philipskurvan}
\begin{itemize}
    \item Samband mellan produktion och prisnivån
    \item Samband mellan arbetslöshet och inflation
\end{itemize}

\subsection{Årlig philipskurva i Sverige}
\begin{figure}[H]
    \centering
    \includegraphics[width=1\textwidth]{philip.png}
\end{figure}

\subsection{BNP}
\begin{itemize}
    \item Marknadsvärdet på alla varor och tjänster som produceras i ett land under en viss tidsperiod
    \item Räknar ej med insatsvaror, dvs. varor som används i produktionen av andra varor
\end{itemize}

Problem: 
\begin{itemize}
    \item Inkluderar inte hemmaproduktion eller svartproduktion
    \item Ta inte hänsyn till miljöförstöring
    \item Att du lagar mat åt dig själv räknas inte, men om du beställer mat räknas det 
\end{itemize}

\subsection{BNP per capita}
BNP per capita är ett välståndsmått men:
\begin{itemize}
    \item Tar inte hänsyn till inkomstfördelning
    \item Inkluderar inte populationens välbefinnande
\end{itemize}

\subsection{HDI -- Human Development Index}
\begin{figure}[H]
    \centering
    \includegraphics[width=0.75\textwidth]{HDI.png}
    \caption{Finns en korrelation mellan BNP per capita och HDI, men det är inte perfekt.}
\end{figure}

\subsection{Egen rapporterad livskvalitet}
\begin{figure}[H]
    \centering
    \includegraphics[width=0.75\textwidth]{life.png}
    \caption{Även egen rapporterad lycka finns det ett samband}
\end{figure}

\subsection{Tillväxt BNP per capita OECDf}


















\end{document}