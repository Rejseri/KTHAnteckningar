\documentclass[12pt,a4paper]{article}
\usepackage[utf8]{}
\usepackage[T1]{fontenc}
\title{Språck sammanfattning}
\author{Jamal Brown}


\renewcommand{\baselinestretch}{0.2}

\setlength{\hoffset}{-18pt}         
\setlength{\oddsidemargin}{0pt} % Marge gauche sur pages impaires
\setlength{\evensidemargin}{0pt} % Marge gauche sur pages paires
\setlength{\marginparwidth}{54pt} % Largeur de note dans la marge
\setlength{\textwidth}{481pt} % Largeur de la zone de texte (17cm)
\setlength{\voffset}{-18pt} % Bon pour DOS
\setlength{\marginparsep}{7pt} % Séparation de la marge
\setlength{\topmargin}{0pt} % Pas de marge en haut
\setlength{\headheight}{13pt} % Haut de page
\setlength{\headsep}{4pt} % Entre le haut de page et le texte
\setlength{\footskip}{27pt} % Bas de page + séparation
\setlength{\textheight}{720pt} % Hauteur de la zone de texte (25cm)

\usepackage{natbib}
\usepackage{graphicx}
\usepackage{amssymb}
\usepackage{amsmath} % Required for the align* environment
\usepackage{float}
\usepackage{pgfplots} 
\usepackage{algorithm}
\usepackage{algpseudocode}
\pgfplotsset{compat=1.18}


\begin{document}
\section{Föreläsning 1: Intro}
\subsection{Vad är makroekonomi?}
Studerar ekonomi som en helhet, inte enskilda marknader.
På både lång och kort sikt.

\subsection{Tre modeller}
\begin{itemize}
    \item Mycket långsikt: Solowmodellen
    \item Långsikt: AS-AD modellen
    \item Kortsikt: IS-LM modellen
\end{itemize}

\subsection{AS--AD modellen}
Aggregerad efterfrågaekurva(AD)
\begin{itemize}
    \item Sammanvägning av alla efterfrågekurvor i ekonomin
    \item Lutar alltid nedåt
\end{itemize}

\noindent Aggregerat utbudskurva(AS)
\begin{itemize}
    \item Sammanvägning av alla utbudskurvor i ekonomin
    \item Beror på tidshorisonten
\end{itemize}

\subsection{Aggregerad utbud--mycket långsikt}
\begin{figure}[H]
    \centering
    \includegraphics[width=0.5\textwidth]{ASLong.png}
\end{figure}
\begin{itemize}
    \item Ackumulering av produktionsfaktorer
    \item Teknisk utveckling
    \item Studerar vad det är som får kurvan att röra sig åt höger
\end{itemize}

\subsection{Aggregerat utbud--långsikt}
\begin{figure}[H]
    \centering
    \includegraphics[width=0.5\textwidth]{ASMedium.png}
\end{figure}
\begin{itemize}
    \item Klassiska modellen
    \item Produktionsfaktorer uttnyttjas fullt ut
    \item Alltid i jämnvikt
\end{itemize}

\subsection{Aggregerat utbud--kortsikt}
\begin{figure}[H]
    \centering
    \includegraphics[width=0.5\textwidth]{ASShort.png}
\end{figure}
Keynesiansk teori:
\begin{itemize}
    \item Priser och löner är trögrörliga (konstanta)
    \item Produktionsfaktorer inte fullt utnyttjade
\end{itemize}

\newpage
\subsection{Dynamik på kort och långsikt}
Exempel: Expansiv penningpolitik
\begin{figure}[H]
    \centering
    \includegraphics[width=1\textwidth]{Dynamic.png}
\end{figure}

\noindent På kort sikt:
\begin{itemize}
    \item Ökar produktionen dvs. kvantiteten
    \item Priserna är konstanta
\end{itemize}

\noindent På lång sikt:
\begin{itemize}
    \item Priserna ökar
    \item Produktionen återgår till det ursprungliga
    \item Ingen effekt på produktionen
\end{itemize}

\noindent På medellång sikt:
\begin{figure}[H]
    \centering
    \includegraphics[width=0.5\textwidth]{asm.png}
\end{figure}
\begin{itemize}
    \item Företaget producerar mer än det naturliga utbudet
    \item Företaget höjer priserna
    \item Produkten återgår till det naturliga utbudet
    \item $A\rightarrow B \rightarrow C$
\end{itemize}

\subsection{BNP och real tillväxt, Sverige 1820--2018}
\begin{figure}[H]
    \centering
    \includegraphics[width=0.75\textwidth]{BNP.png}
\end{figure}

\begin{itemize}
    \item Brukar mätas som procentuell förändring av BNP
    \item Normalt 2--3\% per år
    \item Beror på faktoranvändning och produktivitet 
\end{itemize}

\subsection{BNP och potentiell BNP}
\begin{figure}[H]
    \centering
    \includegraphics[width=0.75\textwidth]{BNP2.png}
\end{figure}

\begin{itemize}
    \item Potentiell BNP: Om samtliga produktionsfaktorer används fullt ut på ett långsiktigt hållbart sätt
    \item Produktionsgap = BNP - Potentiell BNP
    \item Övernyttjande bidrar till inflation
    \item Undernyttjande bidrar till arbetslöshet
\end{itemize}

\subsection{Produktionsgap, Sverige 1945--2018}
\begin{figure}[H]
    \centering
    \includegraphics[width=0.75\textwidth]{BNP3.png}
\end{figure}

\begin{figure}[H]
    \centering
    \includegraphics[width=0.75\textwidth]{BNP4.png}
\end{figure}

\begin{itemize}
    \item Vi kan se att variationen i produktionsgapet blir större och större i nominala termer
    \item Men procentuellt är den i princip lika stor
\end{itemize}

\subsection{Konjunkturcykelns terminologi}
\begin{figure}[H]
    \centering
    \includegraphics[width=0.75\textwidth]{konjunktur.png}
\end{figure}
\begin{itemize}
    \item BNP över potentiell BNP: Högkonjunktur
    \item BNP under potentiell BNP: Lågkonjunktur
    \item BNP påväg upp: konjunkturuppgång
    \item BNP påväg ner: konjunkturnedgång
\end{itemize}

\subsection{Samband produktion och arbetslöshet}
\begin{figure}[H]
    \centering
    \includegraphics[width=0.75\textwidth]{BNP5.png}
\end{figure}

\begin{itemize}
    \item När produktionen ökar, minskar arbetslösheten
    \item När produktionen minskar, ökar arbetslösheten
\end{itemize}

\subsection{AS och Philipskurvan}
\begin{itemize}
    \item Samband mellan produktion och prisnivån
    \item Samband mellan arbetslöshet och inflation
\end{itemize}

\subsection{Årlig philipskurva i Sverige}
\begin{figure}[H]
    \centering
    \includegraphics[width=1\textwidth]{philip.png}
\end{figure}

\subsection{BNP}
\begin{itemize}
    \item Marknadsvärdet på alla varor och tjänster som produceras i ett land under en viss tidsperiod
    \item Räknar ej med insatsvaror, dvs. varor som används i produktionen av andra varor
\end{itemize}

Problem: 
\begin{itemize}
    \item Inkluderar inte hemmaproduktion eller svartproduktion
    \item Ta inte hänsyn till miljöförstöring
    \item Att du lagar mat åt dig själv räknas inte, men om du beställer mat räknas det 
\end{itemize}

\subsection{BNP per capita}
BNP per capita är ett välståndsmått men:
\begin{itemize}
    \item Tar inte hänsyn till inkomstfördelning
    \item Inkluderar inte populationens välbefinnande
\end{itemize}

\subsection{HDI -- Human Development Index}
\begin{figure}[H]
    \centering
    \includegraphics[width=0.75\textwidth]{HDI.png}
    \caption{Finns en korrelation mellan BNP per capita och HDI, men det är inte perfekt.}
\end{figure}

\subsection{Egen rapporterad livskvalitet}
\begin{figure}[H]
    \centering
    \includegraphics[width=0.75\textwidth]{life.png}
    \caption{Även egen rapporterad lycka finns det ett samband}
\end{figure}

\subsection{Tillväxt BNP per capita OECD}
\begin{figure}[H]
    \centering
    \includegraphics[width=1\textwidth]{BNP6.png}
\end{figure}

\subsection{BNP per capita i Sverige 1950--2031}
\begin{figure}[H]
    \centering
    \includegraphics[width=1\textwidth]{BNP7.png}
\end{figure}

\subsection{Sveriges utveckling i OECD välståndsliga}
\begin{figure}[H]
    \centering
    \includegraphics[width=0.75\textwidth]{BNP8.png}
\end{figure}
\begin{itemize}
    \item Sverige har legat i topp i många år
    \item Men nu har vi halkat ner till från 4:e till 13:e plats
\end{itemize}

\subsection{Nationalräkenskaper}
Vad är nationalräkenskaper?
\begin{itemize}
    \item Visar vad BNP består av
    \begin{itemize}
        \item Inkomst
        \item Produktion
        \item Användning
    \end{itemize}
    \item Ett sätt att karaktärisera ekonomin
\end{itemize}

\subsection{Ekonomins storlek förenklat}
\begin{figure}[H]
    \centering
    \includegraphics[width=1\textwidth]{BNP9.png}
\end{figure}

\subsection{Utbudets och efterfrågans komponenter}
Totala utbud av varor och tjänster:
\begin{itemize}
    \item Inhemsk produktion (Y)
    \item Import (Q)
\end{itemize}

\noindent Totala efterfråga på varor och tjänster
\begin{itemize}
    \item Hushållens konsumtion (C)
    \item Företagens investeringar (I)
    \item Offentliga konsumtion (G)
    \item Export (X)
\end{itemize}

\noindent Fundamental nationainkomstidentitet:
{\large $$ Y + Q = C + I + G + X $$}
{\large $$\implies Y = C + I + G + NX$$}
{\large $$NX = X - Q$$}
\noindent Där $NX$ är nettoexporten

\subsection{Hushållens konsumtion (C)}
Inköp av varor hos hushållen 
\begin{itemize}
    \item Varaktiga varor: bilar, möbler, etc.
    \item Icke-varaktiga varor: mat, fika, etc.
    \item Tjänster: frisör, restaurang, etc.
\end{itemize}

\subsection{Hushållens konsumtion (C) \% av BNP}
\begin{figure}[H]
    \centering
    \includegraphics[width=0.75\textwidth]{C.png}
\end{figure}
I Sverige består konsumtionen av BNP betydligt mindre än andra länder. I Sverige har vi väldigt stor offentilg konsumtion istället

\subsection{Företagens investeringar (I)}
\begin{itemize}
    \item Bostadsbyggande
    \item Nya maskiner 
    \item Byggade av favriker och kontor
    \item Förändring av varulager
\end{itemize}

Investeringar inkluderar inte:
\begin{itemize}
    \item Investering i utbildning
    \item Köp av obligationer och aktier
    \item Konsumenternas privata investeringar i t.ex. aktier och fonder räknas som konsumtion, inte investeringar
\end{itemize}

\subsection{Offentlig konsumtion (G)}
\begin{itemize}
    \item Nationellt försvar
    \item Löner av statliga anställda
    \item Bidrag till hushåll och företag räknas inte som offentlig konsumtion, utan som transfereringar
\end{itemize}

\subsection{Olika komponenter i Sverige}
\begin{figure}[H]
    \centering
    \includegraphics[width=0.75\textwidth]{components.png}
\end{figure}
\newpage
\subsection{Nominel vs real BNP}
Nominell BNP beräknas 
$$NGDP_t = \sum_{i=1}^{n} P_i Q_i$$

\noindent Där $P_i$ är priset på vara $i$ och $Q_i$ är kvantiteten av vara $i$ som produceras under år $t$.

\vspace{15pt}
\noindent Tillväxt i nominell BNP beräkas som: 
$$G_{t+1} = \frac{NGDP_t - NGDP_{t-1}}{NGDP_{t-1}}$$
$$G_{t+1} = \sum_{i}\omega_{i,t}(\hat{P}_{i,t+1} + \hat{Q}_{i,t+1} + \hat{P}_{i,t+1} \hat{Q}_{i,t+1})$$
$$\omega_{i,t} = \frac{P_i Q_i}{NGDP_t}$$
$\hat{}$ representerar procentuell förändring mellan år $t$ och $t+1$
\vspace{15pt}

\noindent Hur ska vi fånga upp enbart kvanitetsförändring?
$$G_{t+1} = \sum_{i}\omega_{i,t}(\hat{P}_{i,t+1} + \hat{Q}_{i,t+1} + \hat{P}_{i,t+1} \hat{Q}_{i,t+1}), \omega_{i,t} = \frac{P_i Q_i}{NGDP_t}$$

\noindent Vi håller priserna konstanta vid ett valt basår
$$g_{t+1} = \sum_{i} \hat{\omega}_{i,t} \hat{Q}_{i,t+1}, \hspace{15pt} \hat{\omega}_{i,t} = \frac{P_{i,B} Q_{i,t}}{P_{i,B} Q_{i,B}}, \hspace{15pt} RGDP_t = \sum_{i} P_{i,B} Q_{i,t}$$

\noindent $g$ är den reala tillväxten i ekonomin mellan år $t$ och $t+1$. $P_{i,B}$ är priset på vara $i$ i basåret. $RGDP$ är den reala BNPn.

\vspace{15pt}
\noindent $G$ inkluderar båda pris- och kvanitetsänderingar, medan $g$ inkluderar endast kvantitetsförändringar.

\subsection{Nominell vs real BNP i Sverige}
\begin{figure}[H]
    \centering
    \includegraphics[width=0.8\textwidth]{BNP10.png}
\end{figure}

\subsection{Viktiga identiteter}
\begin{itemize}
    \item Nationalinkomstidentiteten: $Y = C + I + G + NX$
    \item Disponibel inkomst (YD): $YD = Y + TR - TA_{skatter}$
    \item Fördelning av disponibel inkomst: $YD = C + S$
    \item Sparande (S): $S = I + (G+TR-TA)+NX, \hspace{15pt} (G + TR - TA) = BD$
    \item Hushållen kan fördela sitt sparande på tre sätt
    \begin{itemize}
        \item Lån til företag (I)
        \item Lån till staten (BD)
        \item Lån till utlandet (NX)
    \end{itemize}
\end{itemize}

\subsection{Inflation (KPI)}
Tidsserie över KPI i Sverige 1980--2025
\begin{figure}[H]
    \centering
    \includegraphics[width=0.6\textwidth]{kpi.png}
\end{figure}

\subsection{Inflation: Förändering av KPI (YoY)}
\begin{figure}[H]
    \centering
    \includegraphics[width=0.6\textwidth]{kpi2.png}
\end{figure}

\subsection{Inflation: beräkning}
Procentuella förändring i den allmänna prisnivån mellan år $t$ och $t+1$:
$$\pi_{t+1} = \frac{P_{t+1} - P_t}{P_t}$$

\subsection{BNP--deflator}
Definierar som kvoten mellan nominell BNP och real BNP:
\begin{figure}[H]
    \centering
    \includegraphics[width=0.6\textwidth]{deflator.png}
\end{figure}

\subsection{Konsumentprisindex, KPI}
\begin{itemize}
    \item Mäts med en standardiserad varukorg---representativt för konsumenter
    \item Visar levnadskostnaderna för ett genomsnittligt hushåll
\end{itemize}

\subsection{Jämförelse mellan KPI mot BNP--deflator}
\begin{itemize}
    \item KPI är mer begränsad varukorg
    \item BNP--deflator inkluderar alla varor och tjänster som produceras i ekonomin
    \item KPI inkluderar importvaror, medan BNP--deflator inte gör det
\end{itemize}

\subsection{Producentprisindex, PPI}
\begin{itemize}
    \item Standardiserad varukorg för producenter
    \item Fångar produktionskostnad för ett genomsnittligt företag
    \item Inkluderar både råvaror och insatsvaror
    \item Fångar inflation i tidigare skede än KPI
\end{itemize}

\subsection{Inflation, nominell och real ränta}
Antar att jag idag har $x$ kronor då kan jag vid tidspunkt $t$ köpa:
$$\frac{x}{P_t} \text{enheter BNP}$$
\noindent Om jag istället investerar mina pengar kan jag vid tidpunkt $t+1$ köpa: 
$$\frac{x(1+R)}{P_{t+1}} \text{enheter BNP}$$ 
\noindent Den reala räntan fås genom:
$$r_{t+1} = \frac{x(1+R_t)/P_{t+1} - x/P_t}{x/P_t} = \frac{1+R_t}{1 + \pi_{t+1}}$$
$$\implies R_t = r_{t+1} + \pi_{t+1} + r_{t+1} \pi_{t+1}$$

\section{Föreläsning 2: Ekonomisk tillväxt}
\subsection{Inkomstfördelning 2023}
\begin{itemize}
    \item Europa, Nordamerika samt Chile är hög inkomstländer
    \item BNP per capita är 112 gånger högre i top 5 länderna än i botten 5 länderna
    \item BNP per capita och ljusintensitet samvarierar
\end{itemize}

\subsection{Inkomstfördelning 156 länder}
\begin{figure}[H]
    \centering
    \includegraphics[width=1\textwidth]{inkomstfördelning.png}
\end{figure}

\begin{itemize}
    \item Vi kan se ojämnlikheten i BNP per capita har ökat 1970--2014
    \item Däremot ökar BNP per capita i helhet på alla länder 
\end{itemize}

\subsection{Lorenzkurvan}
\begin{figure}[H]
    \centering
    \includegraphics[width=1\textwidth]{lorenz.png}
\end{figure}

\begin{itemize}
    \item Mäter andel av världens inkomst som en funktion av andel av världens population
    \item Gini-koefficienten är arean mellan den räta linjen och Lorenzkurvan, delat med arean under den räta linjen
\end{itemize}

\subsection{Tillväxt takt}
\begin{figure}[H]
    \centering
    \includegraphics[width=0.75\textwidth]{growth.png}
    \caption{Om dessa länder kommer att ha samma tillväxttakt från 1990-2019}
\end{figure}

\subsection{BNP per capita rank 1970 vs 2019}
\begin{figure}[H]
    \centering
    \includegraphics[width=0.85\textwidth]{rank.png}
    \caption{illustration av ländernas rank 2019 vs 1970}
\end{figure}

\subsection{Tillväxtdekomponering}
Produktionsfunktion: $Y = A \cdot F(K,N)$
\begin{itemize}
    \item $Y$: BNP
    \item $A$: Teknologisk nivå
    \item $K$: Kapitalstock
    \item $N$: Antal sysselsatta
\end{itemize}

\noindent I perfekt konkurrens gäller:
$$w = MPL = A \cdot F_N(K,N), \hspace{15pt} r = MPK = A \cdot F_K(K,N)$$

\noindent Sammanfattningvis:
$$\frac{\Delta Y}{Y} = \frac{\Delta A}{A} + \underbrace{(\frac{w \cdot N}{Y})}_{1-\theta}\frac{\Delta N}{N} + \underbrace{(\frac{r \cdot K}{Y})}_{\theta}\frac{\Delta K}{K}$$

\begin{itemize}
    \item Arbetskraftens bidgrag till tillväxt: {\Large $(\frac{w \cdot N}{Y})\frac{\Delta N}{N}$}
    \item Kapitalets bidrag till tillväxt: {\Large $(\frac{r \cdot K}{Y})\frac{\Delta K}{K}$}
\end{itemize}
\newpage
\subsection{Tillväxtdekomponering, empiriskt}
Räkna ut teknologi faktorn i en solow residual:
\vspace{15pt}

\noindent Exempel på tillväxtdekomponering i Sverige 2010--2019:
\begin{figure}[H]
    \centering
    \includegraphics[width=0.75\textwidth]{dekomponering.png}
\end{figure}

\subsection{Tillväxt teori}
Två typer 
\begin{itemize}
    \item Exogen tillväxtteori: Teknologisk utveckling sker utanför modellen
    \item Endogen tillväxtteori: Teknologisk utveckling sker inom modellen
    \item Solowmodellen: En neoklassisk exogen tillväxtmodell
    \item Solowmodellen baseras på en neoklassisk produktionsfunktion 
\end{itemize}

\subsection{Neoklassisk produktionsfunktion, villkor}
Alla produktionsfunktioner måste uppfylla följande villkor för att användas i solowmodellen:
\begin{enumerate}
    \item Konstant skalavkastning: $F(A,\lambda K, \lambda N) = \lambda Y, \text{\hspace{15pt}} \lambda > 0$
    \item Positiva och avtagande marginalprodukter: 
    \begin{itemize}
        \item {\large $\frac{\partial F(A,K,N)}{\partial K} > 0$}
        \item {\large $\frac{\partial F(A,K,N)}{\partial N} > 0$}
        \item {\large $\frac{\partial^2 F(A,K,N)}{\partial K^2} < 0$}
        \item {\large $\frac{\partial^2 F(A,K,N)}{\partial N^2} < 0$}
    \end{itemize}
    \item Inadavillkor
    \begin{itemize}
        \item {\large $\lim\limits_{K \to 0}\frac{\partial F(A,K,N)}{\partial K} = \lim\limits_{N \to 0}\frac{\partial F(A,K,N)}{\partial N} = \infty$}
        \item {\large $\lim\limits_{K \to \infty}\frac{\partial F(A,K,N)}{\partial K} = \lim\limits_{N \to \infty}\frac{\partial F(A,K,N)}{\partial N} = 0$}
    \end{itemize} 

\end{enumerate}

\subsection{Solowmodellen}
Produktionsfunktion: $Y = A \cdot F(K,N)$
\vspace{5pt}

\noindent På grund av konstant skalavkastning kan vi skriva om produktionsfunktionen i per capita termer:
\begin{itemize}
    \item $y = f(k)$
    \item $y = \frac{Y}{N}$
    \item $k = \frac{K}{N}$
    \item $f(k) = A \cdot F(\frac{K}{N},1)$
\end{itemize}

\noindent Aggregerad kapitalachumuleringsfunktion: 
\begin{itemize}
    \item $\Delta K = sY - dK$
    \item Vi antar en sluten ekonomi dvs. en ekonomi utan utrikeshandels 
    \item Vi har inte heller in offentlig sektor, därför går alla besparingar till investeringar
\end{itemize}

\noindent Förändering i kapitalstock är alltså:
$$sparkvot * produktion - deprecieringstakt * kapitalstock$$

\noindent i per capita termer:
$$\Delta k = s y - (d + n)k, \hspace{7pt}\text{n är populations tillväxten}$$

\noindent Sammanfattningsvis två funktioner:
\begin{itemize}
    \item Produktionsfunktion: $y = f(k)$
    \item Kapitalackumuleringsfunktion: $\Delta k = s y - (d + n)k$
    \item Systems jämnvikt: $\Delta k = 0 \implies s f(k^*) = (d + n)k^*$
\end{itemize}

\subsection{Solowmodellen, visuell lösning}
\begin{figure}[H]
    \centering
    \includegraphics[width=1\textwidth]{solow.png}
\end{figure}

\subsection{Solowmodellen, ökar sparande till s'}
Vad händer när politiker gör det mer lönsamt att spara, dvs. ökar sparkvoten från $s$ till $s'$?
\begin{figure}[H]
    \centering
    \includegraphics[width=1\textwidth]{solow2.png}
\end{figure}

\subsection{Hur ser övergången ut?}
\begin{itemize}
    \item $y = f(k)$
    \item $\Delta k = s y - (d + n)k$
\end{itemize}
\begin{figure}[H]
    \centering
    \includegraphics[width=0.6\textwidth]{solow3.png}
\end{figure}

\subsection{Slutsatser}
\begin{itemize}
    \item Ekonomin sträver mot en jämnvikt
    \item Vid jämvikten i den enkla modellen är tillväxten per capita lika med 0
    \item OM sparkvoten ökar
    \begin{itemize}
        \item Tillväxttakten ökar omedelbart och konvergerar sedan mot 0
        \item Om kapitalstocken, sparandet och BNP per capita ökar, så konvergerar över tid mot högre jämnviktsnivåer
        \item Konsumptionen per capita kan kovergera antingen mot en ny högre eller lägre jämnviktsnivå
    \end{itemize}
\end{itemize}

\subsection{Solowmodellen med teknologisk utveckling}
Produktionsfunktion: $Y = F(K, A \cdot N)$

\noindent Omskrivning med konstant skalavkastning:
\begin{itemize}
    \item $\tilde{y} = f(\tilde{k})$
    \item $\tilde{y} = \frac{Y}{A \cdot N}$
    \item $\tilde{k} = \frac{K}{A \cdot N}$
    \item $f(\tilde{k}) = F(\frac{K}{A \cdot N},1)$
\end{itemize}

\noindent Aggregerad kapitalackumuleringsfunktion: 
$$\Delta K = sY - dK$$

\noindent Uttryckt i per capitaform:
$$\Delta \tilde{k} = s \tilde{y} - (d + n + g)\tilde{k}, \hspace{7pt}\text{g är teknologisk utveckling}$$

\subsection{Solowmodellen med teknologisk utveckling, visuellt}
\begin{figure}[H]
    \centering
    \includegraphics[width=1\textwidth]{solow4.png}
\end{figure}

\subsection{Slutsatser: Solowmodellen med teknologisk utveckling}
\begin{itemize}
    \item Ekonomin konvergerar mot ett jämnviktsläge
    \item BNP per effektiv arbetare är konstant vid jämnvikten: $\Delta \tilde{y} = 0$
\end{itemize}

\noindent Vad gäller för BNP per arbetare?
$$\tilde{y} = \frac{Y}{AN} = \frac{y}{A} \implies y = \tilde{y}A$$

\noindent Tillväxttakten ges av:
$$\frac{\Delta y}{y} = \frac{\Delta A}{A} = g$$

\noindent BNP per capital växer i samma takt som teknologin 
\vspace{7pt}

\noindent Tillväxttakten är exogen, förklaras inte i modellen utan är en input parameter

\end{document}