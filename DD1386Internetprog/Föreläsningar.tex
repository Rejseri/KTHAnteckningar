\documentclass[12pt,a4paper]{article}
\usepackage[utf8]{inputenc}
\usepackage[T1]{fontenc}

\renewcommand{\baselinestretch}{0.2}

\setlength{\hoffset}{-18pt}         
\setlength{\oddsidemargin}{0pt} % Marge gauche sur pages impaires
\setlength{\evensidemargin}{0pt} % Marge gauche sur pages paires
\setlength{\marginparwidth}{54pt} % Largeur de note dans la marge
\setlength{\textwidth}{481pt} % Largeur de la zone de texte (17cm)
\setlength{\voffset}{-18pt} % Bon pour DOS
\setlength{\marginparsep}{7pt} % Séparation de la marge
\setlength{\topmargin}{0pt} % Pas de marge en haut
\setlength{\headheight}{13pt} % Haut de page
\setlength{\headsep}{4pt} % Entre le haut de page et le texte
\setlength{\footskip}{27pt} % Bas de page + séparation
\setlength{\textheight}{720pt} % Hauteur de la zone de texte (25cm)

\usepackage{natbib}
\usepackage{graphicx}
\usepackage{float}
\usepackage{xcolor}
\usepackage{minted}

\setminted{
    linenos,
    breaklines,
    frame=lines,
    fontsize=\normalsize,
}
\definecolor{BrickRed}{rgb}{0.8, 0.25, 0.33} % Define BrickRed color

\begin{document}

\section*{Föreläsning 1}
\subsection*{Kretskopplade nätverk}
\begin{itemize}
    \item Förbindelser mellan tå enheter som uprättshålls med växlar innan överföringen kan starta
    \item Kan ta lång tid att upprättahålla en anslutning
    \item Ömtålig, om en resurs går ner så går alla anslutningar som använder resursen ner
    \item Begränsad kapacitet
    \item Exempel: Telefoninätet
\end{itemize}

\subsection*{Paketväxlande nätverk}
\begin{itemize}
    \item Förbindelselös kommunikation
    \item Pakten överförs beroende av varandra
    \item Paketen kan ta olika vägar
    \item Nätet kan inte bli upptaget utan bara långsammare
    \item T.ex IP(Internet Protocol), TCP(Transmission Control Protocol), UDP(User Datagram Protocol)
\end{itemize}

\subsection*{Protokoll}
Protokoll är en samling av relger och överrenskommelser kring kommunikation 
Innehåller bl.a. uppgifter om:
\begin{itemize}
    \item Förbindelser
    \item Vägval
    \item Sönderdelning av data
    \item Sammansättning av data
    \item Upprättande av ordningsföljd vid överföring
    \item Felkorrigering
\end{itemize}

\subsection*{Exempel på HTTP-protokollet}
\begin{figure}[H]
    \centering
    \includegraphics[width=0.8\textwidth]{httpprotocol.png}
    \caption{HTTP Request och Response exempel}
\end{figure}

\subsection*{IP (Internet Protocol)}
\begin{itemize}
    \item Överför data mellan olika nätverk
    \item Paketen kan fördubblas eller försvinna 
    \item IPv4 består av 4 bytes och har formatet xxx.xxx.xxx.xxx där varje xxx är ett tal mellan 0-255
    \item TTL (Time To Live): Kontrollfält
    \item Dagens IPv4 håller på att ersätta IPv6 som har address på 128 bitar
\end{itemize}

\subsection*{Socket}
\begin{itemize}
    \item Ett gränssnitt som används för TCP/IP och UDP/IP och upprätthåller anslutningen baserat på IP för två maskiner
    \item Kod för socket och server finns i föreläsningen
\end{itemize}

\newpage
\subsection*{Sammanfattning}
\subsection*{TCP(Transmission Control Protocol)}
\begin{itemize}
    \item Förbindesleorienterat protokoll
    \item Säkerställer att data kommer fram i rätt ordning
    \item Data förloras och fördubblas inte 
\end{itemize}

\subsection*{UDP(User Datagram Protocol)}
\begin{itemize}
    \item Förbindelselös
    \item Data kan förloras eller fördubblas
    \item Data kan komma i fel ordningsföljd
\end{itemize}

\subsection*{ICMP(Internet Control Message Protocol)}
\begin{itemize}
    \item Kopplat till TCP 
    \item Skickas av mottagare som väntat på ett paket som inte kommit fram
    \item \textit{\large traceroute}
\end{itemize}

\newpage 
\section*{Föreläsning 2}
\subsection*{HTTP protokollet}
\begin{itemize}
    \item Tillståndslös - under inga omständigheter sparas information om tidigare förfrågningar
    \item Utbyggbart - Utvecklarna har möjlighet att lägga till egna metoder och headers för deras specifika tjänster
    \item Hur protokollet funkar:
    \begin{itemize}
        \item HTTP-request skickas från en klient eller webbläsare till en server som en begäran eller förfrågan 
        \item HTTP-repsonse skickas från servern tillbaka till klienten som svar på förfrågan
    \end{itemize}
    \item Sammanfattning av proceduren:
    \begin{enumerate}
        \item Klienten öppnar en anslutning
        \item Klienten skickar en förfråga (request)
        \item Servern skickar en respons (response)
        \item Servern stänger anslutningen
    \end{enumerate}
\end{itemize}

\subsection*{URL (Uniform Resource Locator)}
\begin{itemize}
    \item En resurs är en fil som skickas bit för bit till klienten 
    \item Alternativt en exekverbar fil som ska exekveras hos servern sedan skickas utdata till klienten
    \item De exekvera bara filen konfigureras på så sätt att servern vet att den ska exekvera filen istället för att skicka den som en vanlig fil
\end{itemize}

\subsection*{Standardiserad form för URL}
{\Large {\color{blue}Protokoll}{\color{red}://}värd[:{\color{BrickRed}port}]/sökvägen till resursen}

\subsection*{HTTP-request}
Exempel på HTTP-request:
\begin{figure}[H]
    \centering
    \includegraphics[width=0.8\textwidth]{httprequest.png}
    \caption{HTTP Request exempel}
\end{figure}
\newpage
Generella formatet för en HTTP-request:
\begin{figure}[H]
    \centering
    \includegraphics[width=0.8\textwidth]{httpReq.png}
    \caption{HTTP Request format}
\end{figure}

\subsection*{HTTP-metoder}
Viktiga HTTP-metoder:
\begin{itemize}
    \item GET - Frågar efter en resurs
    \item POST - Skickar data till servern
\end{itemize}

\noindent Mindre viktiga HTTP-metoder:
\begin{itemize}
    \item HEAD - Frågar efter informationen om en resurs
    \item PUT - Skickar document till servern
    \item DELETE - Tar bort en resurs
    \item TRACE - När förfrågan måste passera genom en proxy, gateway etc
    \item OPTIONS - Frågar efter tillåtna (tillgängliga) metoder
\end{itemize}

\subsection*{Viktiga headerfält i en request}
\begin{itemize}
    \item Accept: vilka mime-type av resurs kan hanteras
    \item User-Agent: tex. Mozilla/4.75[en] (Win98)
    \item Host: serverns domän-namn och portnummer
    \item Cookie: används för sessionshantering och upprätthållande av tillstånd
    \item Date: när paketet skapades
\end{itemize}

\subsection*{HTTP-response (binär exempel)}
\begin{figure}[H]
    \centering
    \includegraphics[width=0.5\textwidth]{httpresponse.png}
    \caption{HTTP Response exempel}
\end{figure}
Generella formatet för en HTTP-response:
\begin{figure}[H]
    \centering
    \includegraphics[width=0.8\textwidth]{httpresponsEx.png}
    \caption{HTTP Response format}
\end{figure}

\subsection*{Viktiga headerfält i en response}
\begin{itemize}
    \item Content-Type: mime
    \item Content-Length: storleken på resursen i byte
    \item Server: info om server, t.ex: Apache
    \item Set-Cookie: skickar datan
    \item Set-Cookie: skickar kaka
    \item Date: datum då servern skapade responsen 
\end{itemize}
\newpage
\subsection*{Cookies}
HTTP kan inte skilja klienter åt från varandra därför kan man spara en cookie hos klienten istället.
Servern definierar cookienn i response headern
\begin{itemize}
    \item Set-Cookie: <namn> = <värde>
    \item expires = <datum>
    \item domain = <domän>
    \item path = <bibliotek>
    \item 
\end{itemize}
När servern vid ett senare tillfälle vill skicka en request till samma server så skickas cookien med i request headern:

\subsection*{Response med set-cookie, exempel}
\begin{figure}[H]
    \centering
    \includegraphics[width=0.6\textwidth]{cookieresponse.png}
    \caption{HTTP Response med Set-Cookie header}
\end{figure}

\subsection*{Request med cookie, exempel}
\begin{figure}[H]
    \centering
    \includegraphics[width=0.6\textwidth]{cookierequest.png}
    \caption{HTTP Request med Cookie header}
\end{figure}

\newpage
\subsection*{Set-Cookie Format}
\begin{itemize}
    \item Set-Cookie: <name> = <value>; $\leftarrow$ Raderas när man stänger av webbläsaren
    \item Set-Cookie: <name> = <value>; Expires = <date>; $\leftarrow$ Sparas tills utgångsdatumet
    \item Set-Cookie: <name> = <value>; Domain = <domain-value>;
    \item Set-Cookie: <name> = <value>; Path = <path-value>;
    \item Set-Cookie:: <name> = <value>; Max-Age = <seconds>;
\end{itemize}

\newpage
\section*{Föreläsning 3}
\subsection*{Statuskoder}
\begin{itemize}
    \item 1xx - Information: Kommunkation i protokoll nivå
    \item 2xx - Success: Begäran lyckades
    \item 3xx - Omdirigering: Klienten behöver utföra ytterligare handling 
    \item 4xx - Klientfel: Fel hos klienten
    \item 5xx - Serverfel: Fel hos servern
\end{itemize}

\subsection*{Statuskod 200 (OK)}
\begin{figure}[H]
    \centering
    \includegraphics[width=0.8\textwidth]{statuskod200.png}
    \caption{HTTP Response med statuskod 200}
\end{figure}

\subsection*{Statuskod 303 (omdirigering)}
Servern hittar inte resursen på den angivna platsen och skickar en omdirigeringsrespons till klienten med den nya platsen för resursen i Location-headern.
\begin{figure}[H]
    \centering
    \includegraphics[width=0.8\textwidth]{statuskod303.png}
    \caption{HTTP Response med statuskod 303}
\end{figure}

\newpage
\subsection*{Form.java}
Koden för Form.java finns i canvas
\begin{enumerate}
    \item Läs request 
    \item Kolla om det är GET, POST eller en annan metod
    \item Om GET
    \begin{itemize}
        \item Om cookie inte finns i requesten svara med formulär för att kunna skicka in förnamnet
        \item Om cookie finns hämta data från minnet och avgöra nästa rutt
    \end{itemize}
    \item Om POST
    \begin{itemize}
        \item Läs cookie och hämta tillhörande ata från minnet
        \item Om förnman och efternamnet är null svara med formulär för att kunna skicka in det
        \item Om bara förnamn/efternamn är null svara med final-formulär för att kunna skicka in det saknade namnet och ogilgiltigförklara det namn som redan finns
    \end{itemize}
    \item Om annan metod ignorera
    \item Stäng anslutningen

\end{enumerate}

\subsection*{PRG (Post/Redirect/Get)}
\begin{itemize}
    \item Designmönster
    \item Problem som PRG löser:
    \begin{itemize}
        \item Högre säkerhet vid användning av POST-request
        \item Hindrar oavsiktligt skickande av samma data som skickats i tidigare request
    \end{itemize}
\end{itemize}

\subsection*{Jämförelse av PRG och NO\_PRG}

\begin{figure}[H]
    \centering
    \begin{minipage}{0.45\textwidth}
        \centering
        \includegraphics[width=\textwidth]{prg.png}
    \end{minipage}
    \hfill
    \begin{minipage}{0.45\textwidth}
        \centering
        \includegraphics[width=\textwidth]{noprg.png}
    \end{minipage}
\end{figure}

\newpage
\section*{Föreläsning 4}
\subsection*{HTML (HyperText Markup Language)}
\begin{itemize}
    \item Hypertext: text som med länk till andra dokument
    \item Markup Language: Ett språk som ger instruktioner om hur text och bild ska representeras för användaren
    \item Textkoden består av HTML-element: <b>, <div>, <form, <input>,...
\end{itemize}

\begin{figure}[H]
    \centering
    \includegraphics[width=1\textwidth]{htmlex.png}
    \caption{HTML element exempel}
\end{figure}

\subsection*{Vanliga HTML element}
\begin{itemize}
    \item <html> - html sida
    \item <head> - huvud information
    \item <Htle> - sidans titel
    \item <body> - sidans innehåll
    \item <h1> till <h6> - rubriker
    \item <img source = "jag.jpg"> - bild
    \item <form> <input> - webb formulär
    \item <table><tr><td> - tabell, rad, kolumn
    \item <div> - avdelare för att gruppera element
    \item <p> - paragraf
    \item <span> - gruppera element i rad
    \item <br> - radbrytning
    \item <script> - script språk
    \item <ol> - ordnad lista
    \item <ul> - oordnad lista
\end{itemize}

\subsection*{Generella attribut för HTML}
\begin{itemize}
    \item class - Formatteringsklass för gruppering 
    \item id - referens till specifikt element
    \item style - CSS
\end{itemize}

\subsection*{Vanliga sub-element för HEAD}
\begin{itemize}
    \item <base href 0 "(url för den sidan)"> = Prefix url för alla länkar på den sidan
    \item <link href = "special.css" rel = "stylesheet" type = "text/css"> = Definierar relationen mellan två dokument 
    \item <style type = "text/css"> = Inbäddad CSS, alltid
    \item <script type = text/javascript"> = Inbäddad JavaScript
    \item <title> = Dokumentets titel
    \item <meta> = Information för sökmotorer och webbläsare
\end{itemize}

\subsection*{Dom}
En trädstruktur som representerar HTML-dokumentet, javascript kan:
\begin{itemize}
    \item Lägga till nod
    \item Ta bort nod
    \item Ändra nod innehåll
    \item Visa/dölja noder
\end{itemize}

\subsection*{Formulär och Dom}
I elementet <form> kan man lägga vilka presentatioselement som helst
\begin{itemize}
    \item <input type = "text"> - textfält
    \item <input type = "password"> - lösenordsfält
    \item <input type = "submit"> - submit-knapp
    \item <input type = "radio"> - radioknapp
    \item <input type = "checkbox"> - kryssruta
    \item <select> - rullgardinsmeny
    \item <textarea> - textområde
\end{itemize}
Alla element kan refereras i javascript och ändras dynamiskt genom 
\newline
\noindent window.document.forms[3].efternman.value - refererar till efternamn elementet i det 4:e formuläret på sidan

\subsection*{JavaScript}
JavaScript använder event-handlers för att hantera user-input och andra händelser. Varje event-handler är associerad med ett element i DOM och körs när en specifik händelse inträffar på det elementet. M.h.a JavaScript kan man komma åt HTMl-element i dokumentet genom exempelvis: document.getElementByID("id")

\newpage

\subsection*{Event handler några exempel}
\begin{itemize}
    \item OnBlur - formulärelementet
    \item OnChange - formulärelementet
    \item OnClick - formulärelement, länkar
    \item OnFocus - formulärelementet
    \item OnSelect - text, textarea
    \item OnSubmit - form
\end{itemize}

\subsection*{Exempel på JavaScript}
\begin{minted}{JavaScript}
<script type = "text/javascript">

// Enkel form validering 
function checkAddress(form) {
    // Kontrollera att e-post och telefonnummer är ifyllda
    if (form.email.value.length == 0 || form.email.value.index.indexOf('@') == -1) {
        alert("Du måste ange en giltig e-postadress.");
        form.email.focus();
        return false;
    }
    // Kontrollera att telefonnummer är ifyllt
    if (form.phone.value.length == 0) {
        alert("Du måste ange ett telefonnummer!");
        form.phone.focus();
        return false;
    }
    return true;
}
\end{minted}

\subsection*{CSS - \textbf{C}ascading \textbf{S}tyle \textbf{S}heets}
\begin{itemize}
    \item Används för att styra utseendet på HTML-element
    \item Man slipper upprepningar av massa hårdkodningar i HTML filen
    \item I CSS-filen kan man konfigurera varje element för sig själv.
\end{itemize}

\begin{minted}{css}
/* CSS exempel */
td{
padding-top: 3px; 
padding-left: 5px; 
background-color: "#eeeeee"; 
font-family: "sans-serif"; 
font-size: 8pt;
height: 20px;
border-top: "1px #ffffff
solid"; 
} 

th{ 
background-color: "#ffffff"; 
font-family: "sans-serif"; 
font-size: 9pt;
font-weight: bold; 
height: 22px; 
vertical-align: middle; 
text-align: center; 
padding: 3px; 
border-top: "1px #aaaaaa
dotted"; 
border-bottom: "1px #aaaaaa
dotted"; 
}
    
\end{minted}

\subsection*{Motsvarande HTML fil för CSS-exemplet}
\begin{minted}{html}
<!DOCTYPE html> <html> <head>
<link rel="stylesheet" type="text/css”href="myStyle.css">
</head>
<body>
<table>
<th colspan="2">Post price </th>
<tr> <td>Weight Kg</td> <td>Price$/Kg</td> </tr> 
<tr> <td> 1 </td> <td>100</td> </tr> 
<tr> <td> 2-5 </td> <td>90</td> </tr> 
<tr> <td> 6-10 </td> <td> 80 </td></tr> 
<tr> <td> 10 </td><td> 60 </td></tr>
</table> 
</body></html>
\end{minted}

\subsection*{CSS}
\begin{minted}{css}
p {
    color: red;
    text-align: center;
}
\end{minted}

\begin{itemize}
    \item Selector: p
    \item Declaration: color: red;
    \begin{itemize}
        \item Property: color
        \item Value: red
    \end{itemize}
    \item Declaration: text-align: center;
    \begin{itemize}
        \item Property: text-align
        \item Value: center
    \end{itemize}
\end{itemize}
\subsection*{Variabel deklaration i JavaScript}
\begin{minted}{javascript}
let age = 20;
var age = 20;

if (true) {
    let x = 1;
    var y = 2;
}
// x är inte tillgänglig här, den är block-scoped
// y är tillgänglig här, den är function-scoped

//================================================

// konstant deklaration

const PI = 3.14;

// PI är final och kan inte ändras

\end{minted}
\newpage

\subsection*{Typkonvertering i JavaScript}
\begin{itemize}
    \item 5 * null $\rightarrow$ 0
    \item "5" - 3 $\rightarrow$ 2
    \item "5" + 3 $\rightarrow$ "53"
    \item "5" * 2 $\rightarrow$ NAN
    \item false == 0 $\rightarrow$ true
    \item false === 0 $\rightarrow$ false
    \item "4711" == new String("4711") $\rightarrow$ true
    \item "4711" === new String("4711") $\rightarrow$ false
\end{itemize}

\subsection*{Deklarera funktioner}
\begin{minted}{javascript}
//1. Funktionsdeklaration
const calc = function(x, y) {
    return x + y;
};

//2. Funktionsuttryck
function calc(x, y) {
    return x + y;
}

//3. Arrow-funktion
const calc = (x, y) => {
    return x + y;
};

//4. Kortare Arrow-funktion
const calc = (x, y) => x + y;

// ===============================================
//Optional parameter
    function halvera(a, b) {
        if (b === undefined) {
            return a/2;
        }
        else {
            return a/2;
        }
    }
    halvera(4, 54, "blahonga"); // Ignorerar extra argument
    halvera(4); // Returnerar 2
    halvera(4, undefined); // Returnerar 2
\end{minted}

\subsection*{Högre ordningens funktion}
Nestlade funktioner som kan ta andra funktioner som argument eller returnera en funktion som resultat
\begin{minted}{javascript}

//Ex. 1
function högreOrdningensFunktion(func) {
    return function(x) {
        return func(x) * 2;
    }
}

//Ex. 2
function högreOrdningensFunktion(func) {
    return (a, b) => {console.log(a+b); }
}
\end{minted}

\subsection*{Closure}
Funktioner som returnerar andra funktioner eller tar andra funktioner som argument 
\begin{minted}{javascript}
//1. Funktion som returnerar en annan funktion

function makeMultiplier(multiplier) {
    return function(x) {
        return x * multiplier;
    }
}

//2. Funktion som tar en annan funktion som argument
function applyFunction(func, value) {
    return func(value);
}
\end{minted}

\subsection*{Arrays}
\begin{minted}{javascript}
    let namnlista = ["Alice", "Bob", "Charlie"]; // Skapar en array med tre element
    namnlista.push("David"); // Lägger till "David" i slutet av arrayen
    namnlista.unshift("Eve"); // Lägger till "Eve" i början av arrayen
    namnlista.pop(); // Tar bort det sista elementet ("David")
    namnlista.shift(); // Tar bort det första elementet ("Eve")
    namnlista.length; // Returnerar längden på arrayen (3)
    namnlista[2] = "Petter"; // Ändrar det tredje elementet till "Petter"
    // nytt element på index 100, mellanliggande index blir undefined
    namnlista[100] = "Zara"; 
\end{minted}

\subsection*{Object}
\begin{minted}{javascript}
    let person = {
        name: "Petter",
        labs: [1, 2],
    }

    Object.keys(person); // Returnerar ["name", "labs"]
    Object.assign(person, {labs: [3, 4]}, {betyg: "A"}); // Uppdaterar person-objektet med nya värden
    console.log(person); // ger {name: "Petter", labs: [3, 4], betyg: "A"}
\end{minted}

\subsection*{JSON - JavaScript Object Notation}
\begin{itemize}
    \item Serialiserat format 
    \item Likt JavaScript-objekt och arrays
\end{itemize}
Exempel:
\begin{minted}{json}
{
    "name": "Alice",
    "age": 30,
    "isStudent": false,
    "hobbies": ["reading", "traveling", "coding"],
    "address": {
        "street": "123 Main St",
        "city": "Anytown",
        "country": "USA"
    }
}
\end{minted}




\end{document}
