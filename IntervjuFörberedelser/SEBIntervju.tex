\documentclass[12pt,a4paper]{article}
\usepackage[utf8]{}
\usepackage[T1]{fontenc}
\title{SEB Prep}
\author{Yunshan Luo}


\renewcommand{\baselinestretch}{0.2}

\setlength{\hoffset}{-18pt}         
\setlength{\oddsidemargin}{0pt} % Marge gauche sur pages impaires
\setlength{\evensidemargin}{0pt} % Marge gauche sur pages paires
\setlength{\marginparwidth}{54pt} % Largeur de note dans la marge
\setlength{\textwidth}{481pt} % Largeur de la zone de texte (17cm)
\setlength{\voffset}{-18pt} % Bon pour DOS
\setlength{\marginparsep}{7pt} % Séparation de la marge
\setlength{\topmargin}{0pt} % Pas de marge en haut
\setlength{\headheight}{13pt} % Haut de page
\setlength{\headsep}{4pt} % Entre le haut de page et le texte
\setlength{\footskip}{27pt} % Bas de page + séparation
\setlength{\textheight}{720pt} % Hauteur de la zone de texte (25cm)

\usepackage{natbib}
\usepackage{graphicx}
\usepackage{amssymb}
\usepackage{amsmath} % Required for the align* environment
\usepackage{float}
\usepackage{pgfplots} 
\usepackage{algorithm}
\usepackage{algpseudocode}
\pgfplotsset{compat=1.18}


\begin{document}

\section{Quick view of collateral management}
\subsection{Collateral management}
Collateral management is the process of managing and optimizing the collateral that is used to secure financial transactions. It involves the identification, valuation, and monitoring of collateral assets, as well as the management of collateral agreements and the mitigation of counterparty risk. Effective collateral management is crucial for financial institutions to ensure that they have sufficient collateral to meet their obligations and to minimize the risk of losses in the event of counterparty default.
\subsection{Buy side}
The buy side refers to the part of the financial industry that is involved in the purchasing of securities and other financial instruments. This includes institutional investors such as mutual funds, pension funds, hedge funds, and insurance companies, as well as individual investors. The buy side is responsible for making investment decisions and managing portfolios, with the goal of generating returns for their clients or themselves. They typically work with sell-side firms, which provide research, trading, and other services to facilitate the buying and selling of securities.
\subsection{Sell side}
The sell side refers to the part of the financial industry that is involved in the selling of securities and other financial instruments. This includes investment banks, brokerage firms, and other financial institutions that provide services such as underwriting, market making, research, and trading. The sell side is responsible for facilitating transactions between buyers and sellers, providing liquidity to the markets, and offering investment products and services to clients. They typically work with buy-side firms, which are responsible for making investment decisions and managing portfolios.

\newpage
\section{Repo: Repurchase Agreements}
For collateralized secured loan

\subsection{The steps of repo agreement}
\begin{enumerate}
    \item Security seller sells a security (e.g., a bond) to a security buyer for cash.
    \item The cash is the purchase price of the security.
    \item The agree that on a later date (usually short-term eg. 30 days)
    \item 30 days will be the repo term
    \item After 30 days the seller will repurchase the same security from the buyer at the preagreed price (which is usually higher than the original purchase price to reflect interest).
    \item The interest will be known as the repo rate
    \item The buyer will hold the security during the repo term to make them feel safe
\end{enumerate}

\subsection{Key features of repo agreement}
\begin{itemize}
    \item For legal reasons the collaterized loan is structured as a sale and repurchase agreement, rather than a loan with collateral. 
    \item Repos agreements are usually short term most repos are overnight repos. Longer than 1 day is considered term repo.
    \item Most repos involve high quality securities as collaterals (very liquid, low credit risk, e.g., government bonds)
    \item A collateral may be a specific security or a group of securities (Generall collateral repo)
    \item The interest rate of a general collateral repo rate is called the general collateral rate (GC rate)
\end{itemize}

\subsection{Example question repo rate calculation}
\begin{itemize}
    \item Assume repo rate is 4\% per annum
    \item The purchase price of the security is \$10,000,000
    \item The repo term is 30 days
\end{itemize}
Calculation of the repurchase price:
$$r^{\frac{365}{30}} = 1.04$$
$$r_{month} = 1.04^{\frac{30}{365}} \approx 1.00323$$
$$P_{repurchase} = 1.00323 \times 10,000,000 = 10,032,288$$

\newpage
\subsection{Managing shortfall risk}
The shortfall risk is that the value of the collateral may fall below the loan amount
\begin{itemize}
    \item Use high quality collateral (e.g., government bonds), usually do not fluctuate much in value
    \item Initial margin = $\frac{\text{Security price (security market value)}}{\text{Purchase price (loan amount)}}$
    \item Usually we want the initial margin to be greater than 100\% to provide a buffer against price fluctuations of the collateral
\end{itemize}

\subsection{Example: initial margin}
\begin{itemize}
    \item Intial margin = 106\%
    \item Security price = \$10,000,000
\end{itemize}
Calculation of the purchase price:
$$\text{Initial margin} = \frac{\text{Security price}}{\text{Purchase price}} = 1.06$$
$$\implies P_{purchase} = \frac{P_{security}}{\text{initial margin}} = \frac{10,000,000}{1.06} = 9,433,962$$

\begin{itemize}
    \item If you did a outright sale without buy back obligation it would have been \$10,000,000
    \item Due to the initial margin, the buyer only needs to pay \$9,433,962 for the security, which is less than the security price. This provides a buffer against price fluctuations of the collateral.
\end{itemize}

\subsection{Repurchase price}
When the seller repurchases the security the security will not be sold at market price. 
It would be the purchase price + interest (repo rate) 
% conitnue video 30:30

\subsection{Haircut}
$$\text{Haircut} = \frac{P_{security} - P_{purchase}}{P_{security}}$$
\begin{itemize}
    \item $P_{security} = \$10,000,000$
    \item $P_{purchase} = \$9,433,962$
\end{itemize}
Calculation of the haircut:
$$\text{Haircut} = \frac{10,000,000 - 9,433,962}{10,000,000} = \frac{566,038}{10,000,000} \approx 5.66\%$$

\subsection{Difference between haircut and inital margin}
\begin{itemize}
    \item $P_{security} = Initial margin \cdot P_{purchase}$
    \item $P_{security} = P_{purchase} + \text{Haircut(\%)} \cdot P_{security}$
\end{itemize}

\subsection{Initial margin to haircut}

$$Haircut = 1 - \frac{1}{Initial Margin}$$

\noindent For example, if the initial margin is 106\%, then:
$$Haircut = 1 - \frac{1}{1.06} = 1 - 0.9434 = 0.0566 = 5.66\%$$

\subsection{Example and solution}
\begin{itemize}
    \item Security price = \$100,000,000 (Face value, trading at par for 100\% of face value)
    \item Initial margin = 103\%
    \item Repo rate = 4\%
    \item Repo term = 15 days
\end{itemize}

Purchase price calculation:
$$\frac{\text{Security price}}{\text{Purchase price}} = 1.03$$
$$\frac{100,000,000}{1.03} = 97,087,379$$
$$\implies \text{Purchase price} = 97,087,379$$

Re-purchase price calculation:
$$P_{repurchase} = P_{purchase} + P_{purchase} \cdot \text{repo rate}$$
$$\implies P_{repurchase} = 97,087,379 + 97,087,379 \cdot \frac{0.04 \cdot 15}{365} = 97,246,975$$

\subsection{Ownership}
\begin{itemize}
    \item The seller of the security retains the benefits/risks of ownership
    \item Since they pre-agreed the repurchased the security at the predetermined price regardless of security market value
    \item Income on the bond such as coupons belongs to the security seller
    \item Therefore, the value of the security should taken into consideration by the seller
\end{itemize}

\subsection{Variation margin}
%51:50
%https://www.youtube.com/watch?v=_Z7qZvnhdKE
























\end{document}