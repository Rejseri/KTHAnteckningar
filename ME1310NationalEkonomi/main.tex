\documentclass[12pt,a4paper]{article}
\usepackage[utf8]{inputenc}
\usepackage[T1]{fontenc}
\title{ME1310NationalEkonoomi}
\author{Yunshan Luo}
\date{$\today$}

\renewcommand{\baselinestretch}{0.2}

\setlength{\hoffset}{-18pt}         
\setlength{\oddsidemargin}{0pt} % Marge gauche sur pages impaires
\setlength{\evensidemargin}{0pt} % Marge gauche sur pages paires
\setlength{\marginparwidth}{54pt} % Largeur de note dans la marge
\setlength{\textwidth}{481pt} % Largeur de la zone de texte (17cm)
\setlength{\voffset}{-18pt} % Bon pour DOS
\setlength{\marginparsep}{7pt} % Séparation de la marge
\setlength{\topmargin}{0pt} % Pas de marge en haut
\setlength{\headheight}{13pt} % Haut de page
\setlength{\headsep}{4pt} % Entre le haut de page et le texte
\setlength{\footskip}{27pt} % Bas de page + séparation
\setlength{\textheight}{720pt} % Hauteur de la zone de texte (25cm)

\usepackage{natbib}
\usepackage{graphicx}
\usepackage{float}
\usepackage{amsmath}
\usepackage{amssymb}




\begin{document}

\section*{Föreläsning 1}



\section*{Utbud- och efterfrågemodellen}
\subsection*{Antaganden}
\begin{itemize}
    \item Enskild marknad/bransch
    \item Homogen vara
          \begin{itemize}
              \item Alla produkter är likdana i.e alla kaffekoppar är likadana
          \end{itemize}
    \item Det finns många köpare och säljare på marknaden
    \item Ingen information assymmetri
\end{itemize}



\subsection*{Efterfrågan}
\begin{itemize}
    \item Pris
          \begin{itemize}
              \item Normala varor
                    \begin{itemize}
                        \item Köper mer när priset minskar, köper mindre när priset ökar
                    \end{itemize}
          \end{itemize}
          \begin{itemize}
              \item Giffenvaror
                    \begin{itemize}
                        \item Köper mer när priset ökar, köper mindre när priset minskar
                        \item T.ex pris i låginkomst länder där man köpe mer ris när ris gå upp i pris, \textit{Giffenvaror} är sällsynt i verkligenheten
                    \end{itemize}
          \end{itemize}
\end{itemize}



\subsection*{Typer av varor}
\begin{itemize}
    \item Normala varor
          \begin{itemize}
              \item Köper mer när man har högre inkomst
          \end{itemize}
\end{itemize}
\begin{itemize}
    \item Inferiör vara
          \begin{itemize}
              \item Efterfrågan på varan minskar när inkomsten ökar
              \item T.ex friterad kyckling, vattenmelon och kool aid
          \end{itemize}
\end{itemize}
\begin{itemize}
    \item Substitut
          \begin{itemize}
              \item När Coca Cola blir dyrare ökar efterfrågan på Pepsi
          \end{itemize}
\end{itemize}
\begin{itemize}
    \item Komplement
          \begin{itemize}
              \item Kaffe och mjölk används tillsammans
              \item När priset på mjölk ökar minskar efterfrågan på kaffe
          \end{itemize}
\end{itemize}


\newpage
\subsection*{Efterfråga kurva}
$Q^D = 400 - 20P$ beskriver kvantiteten som en funktion av priset

\begin{figure}[H]
    \centering
    \includegraphics[width=0.75\linewidth]{EfterfrågeKurva.png}
    \caption{Efterfrågan kurva som illustrerar kvantitet som en funktion av pris}
    \label{fig:placeholder}
\end{figure}


\subsection*{Inverterad efterfrågekurva}
$P = 20 - 0.05Q^D$ beskriver hur priset beror på kvantiteten
\begin{figure}[H]
    \centering
    \includegraphics[width=0.75\linewidth]{InverteradEfterfrageKurva.png}
    \caption{Inverterad efterfrågekurva som illustrerar pris funktion av kvantiteten}
    \label{fig:placeholder}
\end{figure}
\newpage
\subsection*{Förändering av andra faktorer än pris}

\subsection*{Inkomsten ökar (på en ej inferiör vara)}
\begin{figure}[H]
    \centering
    \includegraphics[width=0.75\linewidth]{InkomstökningHögerskift.png}
    \caption{Högerskiftet kan motiveras av att inkomsten ökar som driver upp efterfrågan på en ej inferiör vara}
    \label{fig:placeholder}
\end{figure}


\subsection*{Ökad pris på komplement (flingor)}

\begin{figure}[H]
    \centering
    \includegraphics[width=0.75\linewidth]{ÖkadPrisKomplement.png}
    \caption{Vänsterskift tyder på en lägre efterfrågan på grund av prisökningar hos komplement}
    \label{fig:placeholder}
\end{figure}
\newpage

\subsection*{Ökad pris på substitut}

\begin{figure}[H]
    \centering
    \includegraphics[width=0.75\linewidth]{ÖkadprisSubstitut.png}
    \caption{Ökad pris på substitut leder till högre efterfrågan då substitut blir dyrare}
    \label{fig:placeholder}
\end{figure}


\subsection*{Utbudsmodellen}
Utbudsmodellen beror huvudsakligen på dessa faktorer
\begin{itemize}
    \item Pris
    \item Antalet producenter
    \item Produktionskostnader
    \item Produktionsteknologi
    \item Förväntningar om framtidensutveckling
\end{itemize}

\subsection*{Inverterad utbudskurva}
Traditionellt så ritas kurvor med priset (P) i y-axeln och kvantitet (Q) i x-axeln


\begin{figure}[H]
    \centering
    \includegraphics[width=0.5\linewidth]{InverteradUtbudskurva.png}
    \caption{$P = 2 +0.04 \cdot Q^{S}$}
    \label{fig:placeholder}
\end{figure}

\newpage

\subsection*{Minskning av enhetspriset för mjölk}
\begin{figure}[H]
    \centering
    \includegraphics[width=0.75\linewidth]{MinskningEnhetspris.png}
    \caption{Minskning av enhetspriset leder till ett högerskift}
    \label{fig:placeholder}
\end{figure}

\subsection*{Sammanfattning}
Föränderingar i marknadssituationer kan illustreras med vänster- och högerskiften i utbuds och efterfrågekurva

\subsection*{Sammansatt utbuds- och efterfrågekurva}
\begin{figure}[H]
    \centering
    \includegraphics[width=0.6\linewidth]{utbudOchEfterfråga.png}
    \caption{Sammansatt kurva som representerar marknadssituationen}
    \label{fig:placeholder}
\end{figure}
Jämviktspriset representerar av där linjerna möts och kan även räknas ut algebraiskt.

\newpage
\subsection*{Algebraisk uträkning av jämnviktspris}
{\large $$D: P =  20 - 0.05Q^D$$
    $$S: P = 2 + 0.04Q^{S}$$
    \begin{equation}
        \begin{cases}
            P = 20 - 0.05Q^D \\
            P = 2 +0.04Q^S
        \end{cases}
    \end{equation}
    $$E = D \Rightarrow 20 - 0.05Q^D = 2 + 0.04Q^{S} $$
    $$\Rightarrow Q_e = 200, P_e = 10$$}
\subsection*{Utbudsöverskott: Marknad ej i jämnvikt}
$$P>P_e$$
\begin{figure}[H]
    \centering
    \includegraphics[width=0.75\linewidth]{Utbudsöverskott.png}
    \caption{Illustration av utbudsöverskott}
    \label{fig:placeholder}
\end{figure}
\begin{itemize}
    \item Köparna är redo att köpa 100L för 15kr
    \item Producenterna är redo att producera 325L för 15kr
    \item Utbudsöverskottet blir $320 -100 = 225L$
    \item I en oreglerad marknad kommer priset falla mot jämnviktspriset dvs. $15kr/L$
\end{itemize}

\subsection*{Marknadsjämnviktsprocess}
\begin{itemize}
    \item Konsumenterna är villiga att gå med på ett högre pris för att kunna köpa mer
    \item Producenterna är villiga att producera mer för att sälja mer
    \item Processen fortsätter tills jämnviktspriset nås
\end{itemize}

\subsection*{Marknadsjämviktsprocessen}
\begin{itemize}
    \item Beskrivs med två kurvor där båda är en funktion av kvantiteten
    \item Allt förutom priset hålls konstant
    \item Om något som inte är priset föränderas beskrivs det som ett skifte i kurvorna
\end{itemize}

\subsection*{Analysmetod komparativ statik}
\begin{enumerate}
    \item Utgår ifrån jämnvikt
    \item En händelse sker
    \item Tolkar händelsen i form och kurvskiften i utbuds- och efterfråganmodellen
    \item Studera hur jämnviktspriset och kvanititet påverkas
\end{enumerate}

\subsection*{Exempel 1: Komparativ statik}
\begin{itemize}
    \item Händelse:
          \begin{itemize}
              \item En studie visar att mjölk ökar risken för prostatacnacer
          \end{itemize}
    \item Förändering
          \begin{itemize}
              \item Efterfrågekurvan skiftar till vänster
          \end{itemize}
    \item Analys
          \begin{itemize}
              \item Jämnviktspriset faller
              \item Jämviktskvantiteten minskar
          \end{itemize}
\end{itemize}
\begin{figure}[H]
    \centering
    \includegraphics[width=0.95\linewidth]{Ex1.png}
    \caption{Illustration på exempel 1}
    \label{fig:placeholder}
\end{figure}
\newpage

\subsection*{Exempel 2: Komparativ statik}
\begin{itemize}
    \item Händelse
          \begin{itemize}
              \item Regeringen besultar att utöka antalet veterinärkontroller av mjölkböndernas gårdar
              \item Ökad kostnad för bönderna
          \end{itemize}
    \item Förändering
          \begin{itemize}
              \item Utbudskurvan skiftar till vänster
          \end{itemize}
    \item Analys
          \begin{itemize}
              \item Ökad veterinärskontroller innebär ett högre pris på mjölk och minskad mjölkkonsumption
          \end{itemize}
\end{itemize}
\begin{figure}[H]
    \centering
    \includegraphics[width=0.75\linewidth]{ex2.png}
    \caption{Illustration av exempel 2}
    \label{fig:placeholder}
\end{figure}
\subsection*{Komparativ statik: Möjliga utfall}
\begin{figure}[H]
    \centering
    \includegraphics[width=1.05\linewidth]{utfall.png}
    \caption{Möjliga utfall för kurvskiften}
    \label{fig:placeholder}
\end{figure}
\newpage
\subsection*{Elasticitet}
\textbf{Elasticitet} beskriver hur mycket $P$ och $Q$ föränderas givet ett vänster- eller högerskift. Elatsticitet kan även tolkas som hur känslig $P$ och $Q$ är för höger respektive vänsterskift

\subsection*{Priselasticitet för efterfrågan}
{\large $$E^D = \frac{dQ^d / Q^D}{dP / P} = \frac{dQ^D}{dP} \cdot \frac{P}{Q^D}$$}
Förklaring: {\large $$\frac{\text{Procentuella föränderingen i kvantitet}}{\text{Procentuella prisföränderingen}}$$

\subsection*{Exempel: priselasticitet}
$$Q^D = 400 - 20P$$
\vspace{5pt}
$$E^D = \frac{dQ^d / Q^D}{dP / P} =\frac{dQ^D}{dP} \cdot \frac{P}{Q^D} = -20 \cdot \frac{P}{Q^D}$$
\begin{itemize}
    \item $E^D < -1 \implies \text{Elastiskt}$
    \item $E^D = -1 \implies \text{Enhetselastiskt}$
    \item $1 < E^D < 0 \implies \text{Oelastiskt}$
    \item $E^D = -\infty \implies \text{Perfekt elastisk}$
    \item $E^D = 0 \implies \text{Perfekt oelastisk}$
\end{itemize}

\subsection*{Andra typer av elasticitet}
Efterfrågans inkomstelastictet:
\vspace{10pt}
{\Large $$E^D_I = \frac{dQ^D / Q^D}{dI / I} = \frac{dQ^D}{dI} \cdot \frac{I}{Q^D}$$}

\noindent Efterfrågans korspriselasticitet
\vspace{10pt}
{\Large $$E^D_{XY} = \frac{dQ_x^D/Q^D_x}{dP_Y / P_Y} = \frac{dQ^D_x}{dP_Y} \cdot \frac{P_Y}{Q_X^D}$$}

\noindent{\Huge $\square$}

\newpage
\section*{Föreläsning 2}
\section*{Konsument- och producentöverskott}
\begin{itemize}
    \item Visar hur mycket köpare och säljare "tjänar" på att marknaden existerar
    \item Överskott skillnaden visar:
          \begin{itemize}
              \item Hur mycket köpare är villiga att betala och hur mycket de måste betala
              \item Hur mycket säljare får betalt och vad de är villiga att sälja för
          \end{itemize}
    \item Överskottet $\equiv$ $\text{Vad aktören är villig att betala} - \text{Vad aktören behöver betala}$
\end{itemize}

\subsection*{Exempel på konsumentöverskott}
\begin{figure}[H]
    \centering
    \includegraphics[width=0.75\linewidth]{konsumentÖverskott.png}
    \caption{Illustration av konsumentöverskott}
    \label{fig:placeholder}
\end{figure}
\begin{itemize}
    \item Jämnviktspriset är 10kr/L
    \item Konsument A: Gör ett överskott på $19-10 = 9kr$
    \item Konsument B: Gör ett överskott på $15-10 = 5kr$
    \item Konsument C: Gör ett överskott på $10-10=0kr$
    \item Det är ej möjligt för konsumenter att göra negativt överskott
    \item Konsumenter som befinner sig höger om jämnviktspriset deltar inte i marknaden
          \vspace{7pt}
    \item Marknadstotal konsumentöverskott {\large $\equiv \int_{0}^{200} (20-0.05Q^DdQ^D) - 10 \cdot 200$}
    \item Intuition: Konsumenter som är villiga att köpa för över marknadspriset tjänar på att marknaden existerar
\end{itemize}
\newpage
\subsection*{Producentöverskott}
\begin{figure}[H]
    \centering
    \includegraphics[width=0.75\linewidth]{producentöverskott.png}
    \caption{Illustration på producentöverskott}
    \label{fig:placeholder}
\end{figure}
\begin{itemize}
    \item Producentöverskottet är det varje producent får sälja under marknadspriset
    \item Intuition: Producenter som är villiga är sälja under marknadspriset tjänar på att marknaden existerar
    \item Producentöverskott $ = \frac{7 \cdot 200}{2} = 700$
\end{itemize}
\newpage

\subsection*{Marknadens totala överskott}
\begin{figure}[H]
    \centering
    \includegraphics[width=0.75\linewidth]{totalOverskott.png}
    \caption{Illustration av marknadens totala överskott}
    \label{fig:placeholder}
\end{figure}
\begin{itemize}
    \item Marknadens totala överskott $\equiv$ konsumentöverskott + producent överskott
    \item I detta fall, Arean på blå triangel + area på röd triangel
\end{itemize}


\subsection*{Konsumentöverskott när efterfrågan minskar}
\begin{figure}[H]
    \centering
    \includegraphics[width=0.6\linewidth]{vidMinskadEfter.png}
    \caption{Enter Caption}
    \label{fig:placeholder}
\end{figure}
\begin{itemize}
    \item Minskning av efterfrågan är ett vänsterskift av efterfrågekurvan
    \item När detta sker så minskas arean av triangeln vilket ger till mindre konsumentöverskott
\end{itemize}
\newpage
\subsection*{Producentöverskott när efterfrågan minskar}
\begin{figure}[H]
    \centering
    \includegraphics[width=0.75\linewidth]{minskadEfterprod.png}
    \caption{Illustration av förändering av producentöverskottet när efterfrågan minskar}
    \label{fig:placeholder}
\end{figure}
\begin{itemize}
    \item När efterfrågan minskar minskas även arean på den röda triangeln
    \item Det innebär att minskning av efterfrågan implicerar även ett minskad producentöverskott
\end{itemize}

\subsection*{Överskott och elasticitet}
\begin{itemize}
    \item Elasticiteten representeras av lutningen på kurvan
    \item Vi kan se att om lutningen är brantare förändras konsument och producentöverskott mer av kurvskiften
\end{itemize}

\subsection*{Prisregleringar}
\begin{itemize}
    \item Administrativ åtgärd för att begränsa marknadens fria prissättning
          \begin{itemize}
              \item Prisgolv (tex. minimilön)
              \item Pristak (tex. hyresreglering)
          \end{itemize}
    \item Prisgolv gynnar säljaren
    \item Pristak gynnar köparen
    \item Pristak måste vara under jämnviktspriset
    \item Prisgolv måste vara över jämnviktspriset
\end{itemize}

\subsection*{Exempel: Prisgolv på mjölk}
\begin{figure}[H]
    \centering
    \includegraphics[width=1\linewidth]{prisgolv.png}
    \caption{Illustration av prisgolv på mjölk}
\end{figure}
\begin{itemize}
    \item Regeringen vill stötta mjölkbönderna genom att införa ett minimipris på mjölk
    \item Ett prisgolv på 15kr/L
    \item Bönderna är villiga att sälja 325L mjölk för det priset
    \item Konsumenterna är endast villiga att köpa 100L mjölk för det priset
\end{itemize}
\newpage
\subsection*{Analys av prisgolv}
\begin{figure}[H]
    \centering
    \includegraphics[width=1\linewidth]{analysPrisgolv.png}
    \label{fig:placeholder}
\end{figure}
Konsument- och producentöverskottet i ursprungsläget är:
\begin{itemize}
    \item Konsumentöverskott $= A +B + C$
    \item Producentöverskott $= D + E +F$
\end{itemize}
Nya konsument- och producentöverskottet blir:
\begin{itemize}
    \item Konsumentöverskott $=A$
    \item Producentöverskott $= B + D + F$
    \item Dödsviktskostnaden $= D + E$
    \item Slutsats
          \begin{itemize}
              \item Konsumentöverskottet minskar
              \item Producentöverskottet
                    \begin{itemize}
                        \item Producenterna vinner B men förlorar E
                        \item Huruvida producentöverskottet ökar eller minskar beror på lutningen (elasticiteten)
                    \end{itemize}
              \item Samhällsekonomiskförlust = Dödsviktskostnaden
          \end{itemize}
\end{itemize}

\newpage
\subsection*{Kvantitetsregleringar}
\begin{itemize}
    \item Sätter tak på hur mycket som får produceras
          \begin{itemize}
              \item Exempelvis: Fiskekvoter, licenser, importkvoter
          \end{itemize}
    \item Kvantitetsregleringar på mjölkmarknaden
\end{itemize}
\begin{figure}[H]
    \centering
    \includegraphics[width=0.75\linewidth]{kvantitetsreglering.png}
    \caption{Illustration av kvantitetsregleringar}
\end{figure}
\begin{itemize}
    \item Följer samma resonemang för prisgolv
    \item Det nya priset i detta fall blir samma som prisgolvet
\end{itemize}
\newpage
\subsection*{Skatter}
\begin{itemize}
    \item Staten kan använda skatter för att få in pengar och påverka beteende
    \item T.ex punktskatter på cigaretter
\end{itemize}
Antag att staten inför en skatt på 4kr/L för mjölk
\begin{figure}[H]
    \centering
    \includegraphics[width=0.75\linewidth]{mjolkskatt.png}
    \caption{Illustration på vad som händer när staten inför skatt på mjölk}
    \label{fig:placeholder}
\end{figure}
\noindent Konsument- och producentöverskottet i ursprungsläget är:
\begin{itemize}
    \item Konsumentöverskott $= A+B+C$
    \item Producentöverskott $= D+E+F$
\end{itemize}
Konsument- och producentöverskottet i det nya läget blir:
\begin{itemize}
    \item Konsumentöverskott = A
    \item Producentöverskott = F
    \item Staten får D och B
    \item Dödviktskostanderna är den samma dvs. $C$ och $E$
    \item I figuren kommer konsumenterna få stå för $55\%$ av skatten och producenterna får stå för $45\%$
\end{itemize}
Sammanfattning för påverkan av skatter:
\begin{itemize}
    \item Dödsviktskostnader ökar generellt i kvadrat på ehetsskatten
    \item Hur skatter påverkar producenterna och konsumenterna beror på elasticiteten för utbud och efterfrågan
\end{itemize}

\newpage
\subsection*{Fördelning av skattebörda}
\begin{figure}[H]
    \centering
    \includegraphics[width=0.75\linewidth]{skattefordelning.png}
    \caption{Illustration på hur skattebördan fördelas}
\end{figure}
Skatteandelar för konsumenter respektive producenter beräknas:
\begin{itemize}
    \item Konsumenter = $\frac{E^S}{E^S + |E^D|}$
    \item Producenter = $\frac{|E^D|}{E^S + |E^D|}$
\end{itemize}

\subsection*{Subventioner}
Subventioner fungerar som en omvänd skatt och kan artificiellt skapa en högre efterfrågan
\begin{figure}[H]
    \centering
    \includegraphics[width=0.75\linewidth]{subventioner.png}
    \caption{Illustration på vad som händer vid subventioner}
    \label{fig:placeholder}
\end{figure}
\newpage
\section*{Föreläsning 3}
\section*{Konsumenteori}
Vad är det som styr en konsuments beteende. Varför köper en konsument en vara och varför köper konsumenten i just den kvantiteten
\subsection*{Preferenser}
Konsumentens problem är
\begin{itemize}
    \item Bestämma vad och hur mycket och vilken vara som konsumenten ska köpa med pengarna den har
    \item Styrs av preferenser
\end{itemize}
Antagande om preferenser:
\begin{enumerate}
    \item Kompletta: Kan alltid rangordna vad konsumenten föredrar
    \item Desto mer desto bättre
    \item Transitiva: Ex. om jag säger att jag gillar A mer än B och B mer än C så måste jag gilla A mer än C
    \item Avtagande relativt bytesförhållande: Vill alltid byta Coca cola mot chipspåsar om jag hur många burkar cola men inte chips
    \item Transitivitet säger att om jag alltid har varit transitiv så kommer jag vara det för framtida beslut också
\end{enumerate}

\subsection*{Nyttofunktioner}
\begin{itemize}
    \item $U(X)$ tar $X$ är det vi konsumerar och används som argument.
    \item $U(X)$ blir nyttan
    \item Alla tänkbara kombinationer av varor och tjänster är ett möjligt argument
    \item Funktionen är ständigt ökande med avtagande marginalnytta
    \item Ofta kontinuerlig och deriverbar
\end{itemize}



















\end{document}