\documentclass[12pt,a4paper]{article}
\usepackage[utf8]{inputenc}
\usepackage[T1]{fontenc}
\title{SF1626 -- Flervariabelsanalys}
\author{Yunshan Luo}

\usepackage{natbib}
\usepackage{graphicx}
\usepackage[top=1.5cm,left = 2.5cm, right = 2.5cm, bottom= 2.5cm]{geometry}
\usepackage{amssymb}
\usepackage{amsmath} % Required for the align* environment
\usepackage{float}
\usepackage{pgfplots} 
\usepackage{algorithm}
\usepackage{algpseudocode}

\begin{document}

\section{SF1626 -- Flervariabelsanalys}
\section{Koordinat system}
\subsection{Polära koordinater}
\begin{equation}
    \begin{cases}
        x = r \cos \theta \\
        y = r \sin \theta
    \end{cases}
\end{equation}
\textbf{Area elementet:}

\noindent Härleds med hjälp av Jacobian determinant:
\begin{equation}
    \det\begin{bmatrix}
        \frac{\partial x}{\partial r} & \frac{\partial x}{\partial \theta} \\
        \frac{\partial y}{\partial r} & \frac{\partial y}{\partial \theta}
    \end{bmatrix} = 
    \det\begin{bmatrix}
        \cos \theta & -r \sin \theta \\
        \sin \theta & r \cos \theta
    \end{bmatrix} = rcos^2 \theta + r \sin^2 \theta = r(cos^2 \theta + \sin^2 \theta) = r
\end{equation}

$$\implies dA = dxdy = rdrd\theta$$

\subsection{Cylindriska koordinater}
\begin{equation}
    \begin{cases}
        x = r \cos \theta \\
        y = r \sin \theta \\
        z = z
    \end{cases}
\end{equation}

\noindent \textbf{Volym elementet:}
\begin{equation}
    \det\begin{bmatrix}
        \frac{\partial x}{\partial r} & \frac{\partial x}{\partial \theta} & \frac{\partial x}{\partial z} \\
        \frac{\partial y}{\partial r} & \frac{\partial y}{\partial \theta} & \frac{\partial y}{\partial z} \\
        \frac{\partial z}{\partial r} & \frac{\partial z}{\partial \theta} & \frac{\partial z}{\partial z}
    \end{bmatrix} = 
    \det\begin{bmatrix}
        \cos \theta & -r \sin \theta & 0\\
        \sin \theta & r \cos \theta & 0\\
        0 & 0 & 1
    \end{bmatrix} = r
\end{equation}
$$\implies dV = dxdydz = rdrd\theta dz$$

\section{Sfäriska koordinater}
\begin{equation}
    \begin{cases}
        x = \rho \sin \phi \cos \theta \\
        y = \rho \sin \phi \sin \theta \\
        z = \rho \cos \phi
    \end{cases}
\end{equation}

\noindent \textbf{Volym elementet:}
\begin{equation}
    \det\begin{bmatrix}
        \frac{\partial x}{\partial \rho} & \frac{\partial x}{\partial \phi} & \frac{\partial x}{\partial \theta} \\
        \frac{\partial y}{\partial \rho} & \frac{\partial y}{\partial \phi} & \frac{\partial y}{\partial \theta} \\
        \frac{\partial z}{\partial \rho} & \frac{\partial z}{\partial \phi} & \frac{\partial z}{\partial \theta}
    \end{bmatrix} = 
    \det\begin{bmatrix}
        \sin \phi \cos \theta & 0 & -\rho \sin \phi \sin \theta\\
        \sin \phi \sin \theta & 0 & 0\\
        cos\phi & -\rho sin\phi & 0
    \end{bmatrix} = 
    -\rho^2 sin^2\phi cos^2\theta - (-\rho^2 sin^2\phi sin^2\theta) = 
    -\rho^2 sin^2\phi (cos^2\theta + sin^2\theta) = 
    -\rho^2 sin^2\phi
\end{equation}
$$\implies dV = dxdydz = \rho^2 sin\phi d\rho d\phi d\theta$$

















\end{document}